\chapter{Wykorzystywane moduły}
\section{Devise}
Devise to łatwy w użyciu i dostosowaniu do aplikacji moduł rozszerzający Ruby on Rails o proste zarządzanie sesjami. Gwarantuje bezpieczne przechowywanie haseł, kontrolę dostępu, kompletne funkcje logowania, rejestracji, powiadomień mailowych.

\begin{itemize}
\item Napisany jest w rubym
\item Bazuje na silniku RoR, jest całkowicie oparte o strukturę MVC
\item Pozwala na zalogowanie wielu modeli jednocześnie
\item Bazuje na strukturze modułowej - można dołączać tylko te wtyczki, które rzeczywiście potrzebujemy
\end{itemize}

Devise składa się z 10 modułów:

\begin{itemize}
\item Database Authenticatable: koduje i zapisuje zakodowane hasła w bazie danych w celu zapewnienia autoryzacji podczas logowania.
\item Omniauthable: dodaje kompatybilność z mediami społecznościowymi (logowanie przez Facebook, Twitter, Google itp). 
\item Confirmable: wysyła email z potwierdzeniem rejestracji i weryfikuje czy konto jest potwierdzone czy nie.
\item Recoverable: umożliwia resetowanie haseł i wysyła email z instrukcją zresetowania hasła.
\item Registerable: zajmuje się tworzeniem kont użytkowników, także edycji ich profilów i usuwaniem swoich kont. Nie pozwala na edycję innych kont ani na usunięcie nie swojego konta, umożliwia automatyczne zalogowanie po rejestracji
Rememberable: umożliwia zapamiętanie sesji użytkownika w cookies, dzięki czemu pozostaje on zalogowany nawet po zamknięciu przeglądarki.
Trackable: śledzi ilość logowań, adresy IP, rodzaj przeglądarki z której się logowano.
Timeoutable: wylogowuje użytkownika po określonym czasie, jeśli nie stwierdzono żadnej aktywności.
Validatable: dostarcza walidację adresów email oraz haseł. Sprawdza długość hasła, poprawność formatu wprowadzonego adresu email. Jest to opcjonalne i nie przeszkadza w definiowaniu własnych walidacji.
Lockable: blokuje konto po określonej ilości błędnych logowań. Można odblokować konto przez token wysłany na adres email lub po określonym czasie.

\end{itemize}

Devise gwarantuje przechwytywanie przechwytywanie wyjątków, zapewniając bezpieczeństwo oraz generowanie czytelnych komunikatów we wszystkich możliwych przypadkach użycia.


W naszej aplikacji będziemy wykorzystywać następujące moduły devise:

\begin{itemize}
\item database authenticatable
\item registerable
\item recoverable
\item rememberable
\item trackable
\item :validatable
\end{itemize}

\section{ActAsTaggableOn}
ActAsTaggable - to moduł wykorzystywany przez nas do grupowania map oraz użytkowników. Wtyczka ta umożliwia szybkie dodanie funkcji otagowania dowolnego modelu, filtrowanie po tagach, dodawanie wielu tagów jednocześnie czy wyświetlania powiązanej chmury tagów.

Dodatkowe funkcje zapeniane przez ten moduł, to znajdowanie najczęściej i narzadziej używanych tagów, powiązanie wielu rodzajów tagów z jednym modelem (np: tagi jako kategorie, grupy, szkoły - wszystko może być powiązane z modelem użytkownika)

\section{CanCanCan}

CanCanCan to biblioteka do autoryzacji dla Ruby on Rails. Pozwala określić, do jakich zasobów aktualnie zalogowany użytkownik ma dostęp. Wszystkie prawa dostępu są zdefiniowane w pojedynczej lokalizacji (klasie Ability) i nie są powtarzane w kontrolerach, widokach czy zapytaniach do bazy danych.

\section{PG}

Wykorzystywana przez nas baza danych to postgreSQL - wybraliśmy ją, ponieważ serwer Heroku, na którym hostujemy naszą aplikację wykorzystuje tę bazę jako domyślną, co uprościło konfigurację.

Ruby on Rails to framework zaprojektowany do współpracy z bazą danych. Klasa ActiveRecord zawiera zdefiniowany komplet funkcji komunikującej się z bazą, tworzeniem, edycją, czytaniem i usuwaniem rekordów. Jedyne, czego potrzebujemy, by sprawić, by komunikacja z bazą zaczęła działać, to dodanie modułu precyzującego, jaką bazę nasza aplikacja będzie wykorzystywać.

PG - to interfejs dla Ruby umożliwiający współpracę z bazą danych PostgreSql.

\section{Uproszczona składnia}
Aby przyśpieszyć pracę nad aplikacją i ułatwić zarządzanie stylami/skryptami, dzieląc je na wiele plików postanowiliśmy nie wykorzystywać bogatych w niepotrzebne znaki standardowych składni aplikacji webowych. Zainteresowaliśmy się nakładkami na wykorzystywane przez nas języki (HTML, CSS, JavaScript), które po wrzuceniu na serwer produkcyjny są komplikowane i łączone w pojedyncze pliki, by nie generować dziesiątek requestów do serwera.

\begin{itemize}
	\item Slim-Rails - jest to interpreter języka Slim, nakładki na HTML pozwalającej na pominięcie wszystkich tagów zamykających, bazujący na wcięciach (podobnie do składni Python'a).
\item Coffe-Rails - interpreter języka CoffeeScript będącego udoskonaleniem składni JavaScript. Opiera się na tych samych założeniach co Slim czy Sass.
\item Sass-Rails - interpreter języka Sass, będącego nakładką na standardową składnię CSS, usuwający wszelkie nawiasy klamrowe, średniki, umożliwiający definiowanie zagnieżdżonych styli, wprowadzenia zmiennych, czy działań matematycznych.
\item Uglifier - kompressor do assets. Umożliwia minimalizację kompresję plików *.coffee oraz *.sass do zminimalizowanych odpowiedników *.js oraz *.css. Dzięki temu usunięte sa wszystkie białe znaki, a nazwy funkcji zamieniane są na jednoliterowe.
\item Less-Rails - łączy wszystkie pliki assets w jeden, dzieki czemu do serwera idzie jeden request zamiast osobnego requesta na każdy plik ze stylami czy skryptem.
\end{itemize}

\section{TwitterBootstrapRails}
Ta wtyczka umożliwia integrację naszej aplikacji z jednym z najbardziej znanych frameworków frontendowym wyekstraktowanym z Twittera - Bootstrapem.

\section{SimpleForm}
Dodatek umożliwjający łatwe generowanie skomplikowanych formularzy HTML dla dowolnego obiektu.

\section{TurboLinks}
By uniknąć pobierania z serwera skryptób javaScript oraz styli CSS za każdym razem, gdy użytkownik kliknie w link na stronie, używamy turbo linków. Pozwalają one na odświeżenie wyłącznie kodu zawartego między tagami <body></body>, zaś cała sekcja <head></head> jest ładowana tylko raz.

Dzięki temu zabiegowi nasza aplikacja ładuje się wielokrotnie szybciej.

\section{gravastic}
Moduł pozwalający na wyświetlania Gravatarów użytkowników oraz komentujących mapy, bazujac na emailu podanym w formularzu.
